\section{Technology}

Before diving more deeply into this topic, we think that is it wise to present which tools have been used to complete our experiments.

We first had to deal with the issue of machine learning itself, meaning that since we had only a few basics in this domain, we were not really aware of the possibilities that were offered in terms of implementation. Our research process ended up on \cite{ML}, which offers an introduction to machine learning with examples of application in Python. Although called \textit{introduction}, this book offered enough materials to understand the main concepts of predictive models and statistical learning. Moreover, it uses an API called \texttt{scikit-learn} \cite{scikit} which enables a lot of machine learning customisation at a high-level of comprehension, which was perfectly suited to our needs.

Regarding data exhibition, we knew in advance that we would have to deal with a lot of graphics, charts and other statistics. Using the \texttt{Jupyter Notebook} \cite{jupyter} therefore came up as the best way to organise and present our experiments in such a manner that it would be visual enough to have a first interpretation of the results, and also be fairly flexible to modify only a code portion without the need to run the complete script.

Other libraries suggested by \texttt{scikit-learn} have also been used through this work such as \texttt{NumPy} \cite{numpy} for mathematical computations, \texttt{matplotlib} \cite{matplotlib} for graphical representations and \texttt{pandas} \cite{pandas} for datasets creation.