\section{Type of learning}

In \cite{ML}, Andreas C. Müller and Sarah Guido develop two kinds of learning, namely \textit{supervised} and \textit{unsupervised}. The main difference between them resides in the output data. The way supervised algorithms work is as follows: given a set of inputs and corresponding outputs, the algorithm creates a model such that, when given a completely new input, it is able to predict the desired output. Such algorithms are called supervised because some kind of supervision is provided for learning in the form of the desired outputs. On the other hand, unsupervised algorithms are simply not provided with output data. They have to come up with a new model based only on input variables. Since our datasets are built from a combination of 119 features - at least initially, we will see afterwards that gathering or reducing the variables can have a huge impact on performance - and their resulting labels, i.e. packed or not packed, supervised algorithms clearly appear to be more suited to our needs.

The next step in the process of algorithm evaluation is to know whether we are solving a regression or classification problem. Regression is the process of finding a new function that can map the input data to a continuous number within any given range. Prediction of house prices can be seen as a regression problem. In contrast, classification tasks aim at predicting a specific label within a list of predefined possibilities based on provided inputs. Classification can either be \textit{binary}, where the output can only take one out of two values, or \textit{multiclass} where the value of the output is selected from more than two classes. An example of a classification problem could be weather prediction like defining if tomorrow is going to be cloudy, sunny or snowy, while predicting the temperature is more a regression problem.

The objective of the following experiments is to come up with the decision of whether a malware is packed, our problematic thus definitely falls under the category of classification when it comes to machine learning. We could then argue whether the outputs we want to generate belong to a binary or multiclass domain. However, our first focus is to predict the nature of a malware before its packer, our main concern is therefore a binary classification problem. 